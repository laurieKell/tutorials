\documentclass[a4paper,10pt]{article}
%\documentclass[a4paper,10pt]{scrartcl}

\usepackage[utf8]{inputenc}

\title{ICCAT STANDING WORKING GROUP FOR ENHANCING THE DIALOGUE BETWEEN FISHERIES SCIENTISTS AND MANAGERS}
\author{}
\date{}

\begin{document}
\maketitle

\section*{8  Framework for the development of Harvest Control Rules (HCRs)}

...


\subsection*{8.2 Possible process for assessing HCRs, particularly in the context of the development of Management Strategy Evaluations (MSE)}


\subsubsection*{8.2.1 Precautionary Approach}

When managing fisheries decisions often have to be  made with incomplete knowledge the Precautionary Approach (PA) requires that
  \begin{itemize}
     \item Undesirable outcomes should be be anticipated, measures taken to reduce their risk of occurring, corrective measures applied 
     immediately and to be effective within an acceptable time frame. 
     \item \textbf{L}imit and \textbf{t}hreshold \textbf{R}eference \textbf{P}oints used as part of a \textbf{H}arvest \textbf{Control} \textbf{Rule}; and
     \item Consideration must be given to major uncertainties. e.g. status of the stocks relative to reference
points, biology and environmental events.
   \end{itemize}
 
 However, \textbf{HCR}s will not necessarily be precautionary if they are not formally evaluated to determine 
    how well they actually achieve their goals given uncertainty. Therefore at the Third Joint Tuna RFMO meeting (Kobe III) 
    it was recognised that Management Strategy Evaluation (MSE) needs to be widely implemented in the tRFMOs in order to 
    implement a Precautionary Approach for tuna fisheries management. 
 
\subsubsection*{8.2.2 Management Strategy Evaluation}

Management Strategy Evaluation involves the use of simulation modelling to evaluate the impact of the main sources of uncertainty.
Benefits of the approach are 

  \begin{itemize} %[<+->]
     \item It allows a fuller consideration of uncertainty as required by the Precautionary Approach; 
     \item Provides stability if management objectives and how to evaluate how well alternative  
	   management strategies meet them are agreed through a dialogue between scientists and stakeholders; and 
     \item Can be used to guide the scientific process by identifying where the reduction of scientific 
	   uncertainty improve management and so help to ensure that expenditure is prioritised to provide 
	   the best research, monitoring and enforcement. 
  \end{itemize}

\subsubsection*{8.2.3 Process}
  Conducting an MSE requires various steps i.e.
  
  \begin{enumerate} %[<+->]
    \item Identification of management objectives and mapping these to performance measures to quantify how well they are achieved\
    \item Selection of hypotheses about system dynamics.
    \item Conditioning of OMs on data and knowledge and possible rejecting and weighting the different hypotheses.
    \item Identifying candidate management strategies and coding these up as MPs, i.e.the combination of pre-defined data, 
	  together with an algorithm to which such data are input to set control measures.
    \item Projecting the OMs forward using the MPs as feedback control procedures; and
    \item Agreeing the MPs that best meet management objectives.
\end{enumerate}

\subsubsection*{8.2.4 Examples}

Currently there are various initiative being conducted by the SCRS related to MSE, i.e. the development of a Generic MSE
that can be applied to the Albacore and Swordfish stocks in the North and South Atlantic and Mediterranean and the work 
under the GBYP.

  \begin{description}
    \item[Generic MSE] A framework that can be used for highly migratory tuna stocks is being developed.
    This uses an Operating Model conditioned on a range of assumptions about biological processes. The OM can be based either 
    on an existing age based stock assessment, e.g. Multifan-CL for North Atlantic Albacore, or life history
    characteristics for data poor stocks. The Management Procedure is based on a biomass dynamic stock assessment model,
    which is currently being used to provide management advice in the form of the Kobe II Strategy Matrix (K2SM) for 
    Northern and Southern Atlantic stocks of albacore and swordfish, and potentially for the Mediterranean stocks as well.

    \item[Mediterranean Bluefin Tuna] 
    An initial MSE is being developed for Mediterranean Bluefin Tuna. This is intended to identify the
    impact of the main sources of quantified and unquantified uncertainties on management.
    In this work the relative value-of-information for model based and empirical Management Procedures will be compared. 
    This is done by conditioning an
    Operating Model on alternative hypotheses about population and fishery dynamics. Data, fisheries and
    fisheries independent are then sampled from the Operating Model to evaluate different harvest control rules as
    part of a Management Procedure. This allows scenarios and data sets to be simulated that reflect
    uncertainty about our knowledge of biology, ecology and our ability to observe and control the fisheries.
    Different Management Strategies will be evaluated with respect to their ability to meet multiple
    management objectives. This is done by considering the trade-offs between the objectives for different
    choices (e.g. to invest in fisheries independent surveys, tagging studies to estimate natural mortality) and
    the robustness of the MPs, e.g. to environmental variability. This allows the relative benefits of improving
    knowledge on population and fishery dynamics to be evaluated.
  \end{description}

\end{document}
