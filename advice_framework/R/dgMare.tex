\documentclass[a4paper,10pt]{article}
%\documentclass[a4paper,10pt]{scrartcl}

\usepackage[utf8]{inputenc}

\title{}
\author{}
\date{}

\pdfinfo{%
  /Title    ()
  /Author   ()
  /Creator  ()
  /Producer ()
  /Subject  ()
  /Keywords ()
}

\begin{document}
\maketitle

\section*{Description}

This course would introduce DG MARE desk Officers and other officials to the current state of the art in Quantitative Fisheries Science (QFS), as applied at different RFMO scientific bodies, with special emphasis on the solutions used by the various tuna RFMOs.

Current practice in stock assessment, provision of quantitative advice and evaluation through simulation of alternative management plans, will be presented using recent examples.

\section*{Learning goals}

\begin{itemize}
 \item To understand the basic ideas behind the models used by scientists to provide advice of fisheries status and future dynamics.
 \item To identify the comparative advantages and limitations of alternative stock assessment models.
 \item To perceive the difficulties in parameter estimation common to all fisheries models and the effect this has on advice uncertainty.
 \item To understand the ideas behind Management Strategy Evaluation and the difference it brings in the formulation or advice.
 \item To be better equipped for a constructive dialogue with scientists on how Quantitative Fisheries Science can help us managing the fishery system.
\end{itemize}

\section*{Organization}

The different methods and models, and the ideas behind them, would be introduced and

explained, focusing on the following issues:

\begin{itemize}
 \item  Data requirements
 \item  Main assumptions
 \item  Possible limitations
 \item  Interpretation of results
\end{itemize}

Example runs and analyses will be analysed and dissected, with examples taken from recent scientific outputs of different RFMOs.

Participants: Around 10-15  DG MARE staff with involvement in scientific activities in RFMOs. They are familiar with scientific work ongoing In RFMOs but most of them have not been involved in hands-on work for several (lasts) years. They should not be expected to run analysis themselves, but access to relevant software and source code will be given for those willing to replicate and explore them in their own time. The course should be open to other DG MARE staff with scientific interests outside RFMOs.

 
\end{document}
